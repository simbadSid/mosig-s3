
%pattern: loop: computation (large amount of data generated) + write this data.   The write data might represent a significant time overhead and processor stall.


\begin{abstract}
	%************ Context: 
	The high performance computing (HPC) ecosystem is, by design, obsessed with performance optimization.   Developing an HPC-specific application requires the proper performance profiling and analysing tools.
	%************  Need: what we have -> what we want
	The high number of \emph{compute} cores and the complexity of an HPC platform lead these utilities to generate and deal with very large performance-trace files.   %An efficient access to this \notationIO\space resource is mandatory for a reasonable performance-profiling execution time.
	%**********TODO: Task: present custom implementation + serie of enhancements***)
	In this context, we have considered enhancing the \notationIO-access of the \toolTargetSoftware, a state-of-the-art trace-analysis software for HPC executions.
	%************  Object of the document
	We propose an overlapping-\notationIO\space \emph{write} approach to outperform the time-response of the \toolTargetSoftware.
	%************  Findings
	Thanks to a theoretical study of the general pattern followed by the \toolTargetSoftware, we show that our method may bring an improvement up to $75\%$ on this pattern.   We also show that our custom implementation of the \toolTargetSoftware\space allows to reduce significantly the perturbation introduced by overlapping the \emph{write} threads.   Our most enhanced version is thus shown to improve the time-response of the \toolTargetSoftware\space by up to $64\%$.
	%************  Conclusion
%	\\\\
%	**********TODO: Conclusion
	%************  Perspectives
%	\\\\
%	**********TODO: Perspectives

\end{abstract}